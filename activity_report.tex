\documentclass[12pt]{article}
\usepackage[latin1]{inputenc}
\usepackage{url}
\usepackage{pagina}
\usepackage{estilo}

\begin{document}

\title{\vspace{-1.8cm}
  {\Huge{\textbf{Activity Report}}\\
  \Huge{\textbf{A6 - Trustworthy processes}}}
}
\author{\vspace{-0.4cm}
  {Hugo Corbucci and Alfredo Goldman}}
\date{}

\maketitle

\section{Introduction}

The work described here is an effort to reinforce the relationship
between agile methods and Open Source software development. Our main
goal is to provide better strategies and practices for open source
projects with the adopt more agile principles. This is motivated by
the quality of software being produced using agile methods. The focus
of these methodologies is to produce working software quickly and
frequently. And then to grow it up maintaining the quality and
avoiding the chance to introduce bugs with automated tests among other
techniques. With the introduction of the right practices from agile
methods in open source communities, they would surely improve the
quality of the delivered software therefore sustaining its
trustability.

This article is divided in four sections. Section \ref{sec:work}
describes the ongoing work in order to increase agile adoption in open
source communities. Section \ref{sec:problems} describes the
difficulties found during this work.  The next one (Section
\ref{sec:deviations}) provides information regarding what diverted
from the original work plan and why. Finally, Section
\ref{sec:corrections} presents actions taken to correct the direction
of the work according to the knowledge acquired during this work.

\section{Description of work performed}
\label{sec:work}

Three main activities are being developed in this work. The first one,
described in section \ref{sub:research}, regards a research to be done
relating agile methods and open source development. The second one is
an experiment with an open source project being conducted in the
academic environment as shown on Section \ref{sub:experiment}. And the
last one, section \ref{sub:talks} is about dissemination of knowledge
in conferences.

\subsection{Researches to relate open source and agile methods}
\label{sub:research}

Two questionnaires were elaborated to determine how much knowledge
open source practitioners and enthusiasts have regarding agile methods
and vice versa. Each of those were meant to be distributed in the most
important conferences of each community.

The first conference, FISL (\textit{F�rum Internacional de Software
  Livre} - International Forum of Software \textit{Libre}), happened
in Porto Alegre, Brazil, in late April (17th, 18th and 19h) and
gathered over 7000 people from the Open Source Community. The second,
Agile 2008, will take place in August (4th to 8th) in Toronto, Canada,
and, although it is not such a wide event, it will gather a good
number of people from the Americas that are involved with Agile
methodologies. One must understand that FISL's participants count with
lots of users and non contributors supporters whereas Agile 2008 only
attracts really involved people which explains why the Agile 2008 is
much smaller.

The research was elaborated to be distributed certain talks during the
event at the entrance of the room and collected at the exit along with
the talk evaluation form. The FISL questionnaire focuses on how much
knowledge attendees have on agile methods and practices, and how much
of it they apply to their open source projects. The answer would
provide a truthful basis to create tools that could fit the
developers' needs and working methods.

The Agile 2008 questionnaire was more interested in discovering on
what extent the practitioners are enrolled in open source projects;
from those experiences, what they learned from open source
contributions that could be applied to their agile teams. What agile
practices they introduced in their open source communities and what
was the feedback about it. Those answers would provide an insight to
what are the the open source community needs in order to adopt more
agile practices. It would also ensure that the tools can attack the
most common problems faced by open source developers when asked to be
agile.

\subsection{Experiment to compare the products created by an agile
  team and an open source team}
\label{sub:experiment}

In parallel, an open source project
(Archimedes\footnote{http://www.archimedes.org.br/} - The Open CAD) is
hosting an experiment to determine what are the main differences in
the results created by an agile team and a traditional open source
team. Two parallel teams developing almost independently are working
on the project to improve it. One was composed of an experienced coach
with deep knowledge of the system and computer science students
working in a traditional agile environment. The other one, was leaded
by the same coach but with a distributed environment with a more
traditional open source approach for development. The study will last
until July 2008 when the agile team cease work.


\subsection{Dissemination of knowledge in those communities}
\label{sub:talks}

At last, two talks were presented at FISL related Qualipso
Project. The first one was an initiative to introduce agile practices
in open source communities by creating \textit{Coding Dojos}
(\cite{DaveThomas}, \cite{Bossavit}) in those
environments. \textit{Coding Dojos} are meetings in which participants
share knowledge to develop code using Test Driven Development (TDD) to
support their code proposals. It has been very effective in
distributing knowledge amongst participants and improve the habit of
creating automated tests for the software being created.

The second one was to promote Qualipso and the Competence Center that
is being created in S�o Paulo. It presented the main goals of the
entity, where and how the open source community could contact it and
benefit from the center and how it should evolve in the next months.

\section{Difficulties}
\label{sec:problems}

Due to operational problems, the FISL organizers where not able to
distribute our research. However, they will promote an on line
research through the list of presenters. This questionnaire is being
developed and will have to aim for a broader audience to compensate
the lack of precision that would have been obtained with FISL's
audience.

As we could not clarify yet which are the main issues regarding the
adoption of agile methods into open source traditional development
process, it has been hard to start developing tools that may
facilitate such task.

\section{Deviations}
\label{sec:deviations}

The original intention for the Agile study was to have the research
going in an European Agile event. The main events are Agile Open and
XP which will happen in Utrecht, Netherlands and Limerick, Ireland
respectively. Both will happen in the first part of June 2008. Since
the research would not be ready and promoted early enough to be
presented at these conferences, it was decided that it would been done
at Agile.

Instead of having a conference research (FISL's one), another one will
be produced aiming to be answered by open source developers. The
dissemination of this research will happen through open source portals
such as Sourceforge. It will consist of an on line form on which a
statistical analysis will be performed. This analysis should present
groups of open source developers related to the knowledge they posses
and use about agile methods.

Instead of having several unconnected tools as in the initial plan, to
be more flexible to changes, proposals are to have a centralized web
interface that serves as a Facade to the various tools plugged into
that central interface.

\section{Corrective Actions}
\label{sec:corrections}

To adapt to the changes, on line versions of all questionnaires are being
made. They will provide some sort of data validation to ensure that
the answers provided are valid in the sense that they were not filled
by some sort of robot. Researches will also be open for a long period
of time (probably around 2 or 3 months) to ensure a wide reach.

In order to ensure a good statistical analysis, questions will be
added to help creating profile groups of answers. It should allow to
classify answers according to their knowledge regarding agile
methodologies and how they use it in their open source projects. This
can provide a good estimation of agile practices penetration in those
communities.


\bibliographystyle{plain} \bibliography{./biblio}

\end{document}
