\documentclass[12pt]{article}
\usepackage[latin1]{inputenc}
\usepackage{url}
\usepackage{pagina}
\usepackage{estilo}

\begin{document}

\title{\vspace{-1.8cm}
  {\Huge{\textbf{Activity Report}}\\
  \Huge{\textbf{A6 - Trustworthy processes}}}
}
\author{\vspace{-0.4cm}
  {Hugo Corbucci and Alfredo Goldman}}
\date{}

\maketitle

\section{Introduction}

The work described here is an effort to reinforce the relationship
between agile methods and Open Source software development. Our main
goal is to provide better strategies and practices for open source
projects with the adopt more agile principles. This is motivated by
the quality of software being produced using agile methods. The focus
of these methodologies is to produce working software quickly and
frequently and then to grow it up maintaining the quality and
avoiding the chance to introduce bugs with automated tests among other
techniques. With the introduction of the right practices from agile
methods in open source communities, they would surely improve the
quality of the delivered software therefore sustaining its
trustworthiness.

This article is divided in four sections. Section \ref{sec:work}
describes the ongoing work in order to increase agile adoption in open
source communities. Section \ref{sec:problems} describes the
difficulties found during this work.  The next one (Section
\ref{sec:deviations}) provides information regarding what diverted
from the original work plan and why. Finally, Section
\ref{sec:corrections} presents actions taken to correct the direction
of the work according to the knowledge acquired during this work.

\section{Description of work performed}
\label{sec:work}

Three main activities are being developed in this work. The first one,
described in Section \ref{sub:surveys}, regards a research to be done
relating agile methods and open source development. The second
(Section \ref{sub:tools}) one is the development of a set of tools to
increase agile practices adoptions in open source project. And the
last one, Section \ref{sub:talks} is about dissemination of knowledge
in conferences.

\subsection{Suverys to relate open source and agile methods}
\label{sub:surveys}

Two questionnaires were elaborated to determine how much knowledge
open source practitioners and enthusiasts have regarding agile methods
and vice versa. Each of those were meant to be distributed in the most
important conferences of each community.

Regarding the open source community, FISL (\textit{F�rum Internacional
  de Software Livre} - International Forum of Software \textit{Libre})
is the biggest event and the next edition (10.0) will happen in Porto
Alegre, Brazil, in late June 2009 (25th, 26th and 27th). The
organization expect to gather over 8000 people from the Open Source
Community. Work has already been done to ensure the survey to be
included in the registration form.

Regarding the agile community, there are two options. The first one,
XP 2009, will take place in the end of May 2009 (26th to 30th) in
Sardinia, Italy. It is the most important and oldest European
conference for Agile methods and is focused in eXtreme Programming
(XP). The second option is Agile 2009 that will happen from August
24th to 28th in Chicago, USA. It is the main event regarding Agile
methods in the world but attracts mostly north Americans. Both options
are being considered and one will be chosen according to further
discussions with each organization.

Each event should receive a different survey since the public is
supposed to be mostly different. The surver to be distributed among
open source enthusiasts tries to determine how much knowledge do they
have about agile methods and how much of that knowledge do they
currently apply to their projects. Finally, we will try to determinate
which practices lack essential tools to determine if more tools should
be developed.

The questionnaire aimed for agile practicioners is more interested in
discovering on what extent the practitioners are enrolled in open
source projects; from those experiences, what they learned from open
source contributions that could be applied to their agile teams. What
agile practices they introduced in their open source communities and
what was the feedback about it. Those answers would provide an insight
to what are the the open source community needs in order to adopt more
agile practices. It would also ensure that the tools to be developed
really attack the most common problems faced by open source developers
when asked to be agile.

\subsection{Tools to improve agile practices that boost communication}
\label{sub:tools}

The main issue when dealing with open source projects is to handle a
distributed team of volunteers. Although the agile methods community
have been working on distributed agile solutions, the main tools used
to support it are distributed pair programming tools and screen
sharing systems. Those intend to improve communication between two
persons in real-time.

Open source projects' teams tend to work asynchronously and require
improvement in tools that could help this sort of communication
boost. To elaborate such tools, the work was based on agile practices
that focus on information spreading.

First, based on the information workspace practice, one has to
consider that an open source project's workspace is the project's web
site. It means a developer should be able to log in the web site and
discover what is the work being done, what has already been done, what
is the status of the build and test results and a few other
information. Most of this data can be collected automatically from the
source code repository and needs to be presented according to the
team's feeling. Developing such tool and build a set of metrics in it
as a plug in for a forge system would allow projects to easily benefit
from this data with little work.

The second tool comes from the need to plan and organize a release
using elements that are easily moveable and editable. Agile teams tend
to use what they call story cards in a wall to quickly arrange their
time schedule and priorities. Since open source projects frequently
use bug tracking systems to log every requirements (including new
features and bug fixes), leaders should have a way to easily overview
and rearrange the entries in this system. This work could even be
performed partially by the community itself using organizational
systems such as wikipedia.

In order to improve their work flow, agile teams often have
retrospective meetings to evaluate what happend and suggest
improvements. This sort of analysis is rare in open source communities
because it requires all members of the team to be online in a
communication channel at the same time. Logging of the past is kept in
mailing lists or in the repository commits' comments. It does not
allow, however, to have problems that do not regard the code itself
mentioned and solved. If a team had a tool that allowed them to post
comments or notes in a time line over time, the discussions could flow
with the release and the leader could compile suggested approaches to
problems in the mailing list or, automatically, from the website.

Finally, to help keeping every member of the team up to date with the
changes other people are performating, teams should have a tool
related to that informative workspace and to the source code
itself. This tool could be an IRC or Instant Message bot that helps
members of the team communicate and logs chats as well as notes and
comments regarding a story or a commit. It would reduce time required
from the developers to log important messages and important
information would be flashed to the people involved quickly reducing
their work to search for it.

\subsection{Dissemination of knowledge in those communities}
\label{sub:talks}

At last, two talks were presented at FISL 9.0 related Qualipso
Project. The first one was an initiative to introduce agile practices
in open source communities by creating \textit{Coding Dojos}
\cite{DaveThomas,Bossavit} in those environments. \textit{Coding
  Dojos} are meetings in which participants share knowledge to develop
code using Test Driven Development (TDD) to support their code
proposals. It has been very effective in distributing knowledge
amongst participants and improve the habit of creating automated tests
for the software being created.

The second one was to promote Qualipso and the Competence Center that
is being created in S�o Paulo. It presented the main goals of the
entity, where and how the open source community could contact it and
benefit from the center and how it should evolve in the next months.

\section{Difficulties}
\label{sec:problems}

The surveys were intended to be run this but contacts with the
conferences were made too late and led to the impossibility to have it
spread to the atendees. An alternative survey on paper was printed but
the organization had technical problems and was not able to recover
the answers. This time, we are planing way before schedule to try
avoiding such problems. If the surveys do not get inserted in the
registration process, external web sites will be created and we will
try to have the link sent to all attendees.

Finally, choosing a target platform to develop the tools is a problem
since there is a growing amount of free software forges that focus on
different aspects. So far, work is being performed to define what is
the best interface and working system for those tools on a separate
platform using Ruby on Rails. Its integration with those forges is
planned to happen around march 2009.

\section{Deviations}
\label{sec:deviations}

The original intention for the Agile study was to have the research
going in an European Agile event. The main events are Agile Open and
XP which will happen in Utrecht, Netherlands and Limerick, Ireland
respectively. Both will happen in the first part of June 2008. Since
the research would not be ready and promoted early enough to be
presented at these conferences, it was decided that it would been done
at Agile.

Instead of having a conference research (FISL's one), another one will
be produced aiming to be answered by open source developers. The
dissemination of this research will happen through open source portals
such as Sourceforge. It will consist of an on line form on which a
statistical analysis will be performed. This analysis should present
groups of open source developers related to the knowledge they posses
and use about agile methods.

Instead of having several unconnected tools as in the initial plan, to
be more flexible to changes, proposals are to have a centralized web
interface that serves as a Facade to the various tools plugged into
that central interface.

\section{Corrective Actions}
\label{sec:corrections}

To adapt to the changes, on line versions of all questionnaires are being
made. They will provide some sort of data validation to ensure that
the answers provided are valid in the sense that they were not filled
by some sort of robot. Researches will also be open for a long period
of time (probably around 2 or 3 months) to ensure a wide reach.

In order to ensure a good statistical analysis, questions will be
added to help creating profile groups of answers. It should allow to
classify answers according to their knowledge regarding agile
methodologies and how they use it in their open source projects. This
can provide a good estimation of agile practices penetration in those
communities.


\bibliographystyle{plain} \bibliography{./biblio}

\end{document}
